\chapter{Fortran90の基礎3〜プログラムフローの構成〜}
この章では、``条件分岐"という枠組みを学習する。
各シーンでの計算内容をコンピュータに判断させることにより、
より複雑な数値計算を行わせることができる。

\section{条件分岐}
{\ttfamily if}文を使えば, 条件に応じて処理を分岐することができる.
以下は, 実数xを読み込み, もしもその符号が負であれば,
符号を反転する(すなわち絶対値を計算する)プログラムである.
\lstinputlisting[caption={if文の例. }, label=ifstatement1]{source/2/IfStatement1.f90}
{\ttfamily if}文の括弧の中には以下の表のように, 適当な等式, あるいは不等式を書く.

\begin{table}[h]
  \caption{等式, 不等式の書き方. }
  \begin{center}
    \begin{tabular}{ccc}
      \hline
      Fortranでの書き方   & \multicolumn{2}{c}{意味} \\ \hline
      x $==$ y   & x $=$ y &(イコール)\\
      x $/=$ y   & x $\ne$ y &(ノットイコール)\\
      x $<$ y   & x $<$ y &(小なり)\\
      x $<=$ y  & x $\le$ y &(小なりイコール)\\
      x $>$ y  & x $<$ y &(大なり)\\
      x $>=$ y   & x $\ge$ y &(大なりイコール)\\ \hline
    \end{tabular}
  \end{center}
\end{table}


以下は二次方程式の判別式を計算して, それが正, 零, 負であるときにそれぞれ異なる処理をするプログラムである.
\lstinputlisting[caption={判別式による二次方程式の解の判定. }, label=ifstatement2]{source/2/IfStatement2.f90}

\subsection*{$<$演習課題$>$}
二次方程式の判別式を計算し, それが正のときは二実数解,
零のときは重解, 負のときは複素数解を計算するプログラムを作成せよ.

\subsection*{$<$演習課題$>$}
自然対数の底$e$の近似値を以下に示す二つの数列を用いて計算せよ.
$n=1, 2, \cdots$と大きくしていき, 真値へと漸近する様子を確認せよ.
また, 真値との相対誤差が$10^{-10}$以下になったときにプログラムを停止するようにせよ.
\begin{equation}
a_n= \Big( 1+\frac{1}{n}\Big)^n
\end{equation}
\begin{equation}
b_n=\sum_{m=0}^{n}\frac{1}{m!}
\end{equation}

以下に、ソースコードの一部をヒントとして示す。必要であれば参考にして良い。
%TODO a_n を求めるコードのヒントを追加する。

%\begin{equation}
%e=\lim_{n\to \infty} a_n = \lim_{n\to \infty} b_n.
%\end{equation}

%TODO もっと演習課題を追加する。
